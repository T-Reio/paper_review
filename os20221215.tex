\documentclass[dvipdfmx,12pt]{beamer}
\usepackage{lipsum}
\usetheme{Verona}
\usepackage{bxdpx-beamer}
\usepackage{pxjahyper}
\usepackage{minijs}
\usepackage{mathpazo}
\usepackage{amsmath,amssymb}
\usepackage{graphicx}
\graphicspath{{fig_tab/os20221215/}}
\usepackage{array}
\usepackage{tikz}
\usepackage{wrapfig}
\usepackage{float}
\usepackage{here}
\usepackage{lscape}
\usepackage{ascmac}
\usepackage{siunitx}
\usepackage{tabularx}
\renewcommand{\kanjifamilydefault}{\gtdefault}
\hypersetup{% hyperrefオプションリスト
 setpagesize=false,
 bookmarksnumbered=true,%
 bookmarksopen=true,%
 colorlinks=true,%
 linkcolor=blue,
 citecolor=blue,
 urlcolor = magenta
}
\setbeamertemplate{navigation symbols}{}

\title[Currie, Voorheis, and Walker, Forthcoming]{What Caused Racial Disparities \\ in Particulate Exposure to Fall?}
\subtitle{New Evidence from the Clean Air Act \\ and Satellite-Based Measures of Air Quality}
\author[R.Tanji]{Currie, Voorheis, and Walker (AER, Forthcoming): Reviewed by R. TANJI}
\date[12/15/2022 OS Semi.]{December 15th, 2022 \\ Ohtake-Sasaki Seminar}
\institute[]{Osaka University, Graduate School of Economics}

\begin{document}

\begin{frame}\frametitle{}
\titlepage
\end{frame}

\section{Introduction}

\begin{frame}\frametitle{Abstract}
  \begin{itemize}
    \item This paper examines the underlying structure that causes racial differences in exposure to ambient air pollution in the United States.
    \begin{itemize}
      \item The difference have declined significantly over the past 20 years.
    \end{itemize}
    \item Clean Air Act (CAA) explains the excess convergence in Black-White pollution exposure
    \begin{itemize}
      \item Areas with larger Black populations saw greater CAA-related declines in PM2.5 exposure
      \item Over 60\% of the reduction in the racial convergence in PM2.5 pollution exposure since 2000
    \end{itemize}
  \end{itemize}
\end{frame}

\frame{\tableofcontents}

\frame{\sectionpage}
\begin{frame}{Motivation \& Literature}
  \begin{itemize}
    \item The existing evidence about racial disparities in pollution exposure is largely piecemeal and indirect.
    \begin{itemize}
      \item Low income and/or racial minorities in the U.S. have been exposed to environmental burdens (Office, 1983; Chavis and Lee, 1987)
      \item Lack of monitoring device to track small particulates (Fowlie, Rubin, and Walker, 2019)
      \item Alternative measurement: distance to a polluting facility
    \end{itemize}
    \item Moreover, we know very little about why racial gaps in pollution exposure may have changed over time.
  \end{itemize}
\end{frame}

\begin{frame}{This Paper}
  \begin{itemize}
    \item Data: newly available national data on PM2.5 exposure from 2000 to 2015
    \begin{itemize}
      \item 1km-grid measures of ambient air pollution levels for the entire United States
    \end{itemize}
    \item Analyses
    \begin{enumerate}
      \item Document racial gaps in ambient exposure to PM 2.5 and the time-series changes between 2000 and 2015.
      \item Explain the gaps by differences in individual and/or neighborhood characteristics.
      \item Explore the contribution of changes in \textbf{relative mobility} and \textbf{relative improvements} in neighborhood air quality.
      \item Use quantile regression to see the impact of the Clean Air Act and National Ambient Air Quality Standards (NAAQS).
    \end{enumerate}
  \end{itemize}
\end{frame}

\begin{frame}{Summary of Results}
  \begin{enumerate}
    \item African Americans tend to live in the most polluted areas nationally, but the gap has been closing.
    \begin{itemize}
      \item Mean gap in pollution exposure: \SI[per-mode=symbol]{1.5}{\micro \gram \per \cubic \meter} $\rightarrow$ \SI[per-mode=symbol]{0.5}{\micro \gram \per \cubic \meter}
    \end{itemize}
    \item differences in individual or household-level characteristics such as income, explain only a tiny part of observed convergence in pollution levels.
    \begin{itemize}
      \item relative mobility differences or changes in Black-White population shares are not able to explain the observed convergence in pollution exposure
    \end{itemize}
    \item Much of this improvement of air quality around African Americans' is driven by the introduction of the PM2.5 NAAQS.
    \begin{itemize}
      \item Spatially targeted nature of the CAA regulations contributes to the observed convergence in
      mean PM2.5 differences between Blacks and Whites.
    \end{itemize}
  \end{enumerate}
\end{frame}

\begin{frame}{Contributions}
  \begin{enumerate}
    \item The first paper to link national representative survey to nationwide grid of PM 2.5 mesurement.
    \begin{itemize}
      \item Explored the causal determinants of narrowing pollution gaps between racial groups over time.
      \item Explore how much variation in pollution exposure be explained by individual endowments (income), aggregate neighborhood-level (average years of schooling) characteristics.
      \item External validity (the spatially continuous PM2.5 measurements)
    \end{itemize}
  \end{enumerate}
\end{frame}

\begin{frame}{}
  \begin{enumerate}
    \setcounter{enumi}{1}
    \item Effects of environmental policy
    and the Clean Air Act more specifically
    \begin{itemize}
      \item Previous literature estimates average effects of policies (Chay and Greenstone, 2003; Isen, Rossin-Slater, and Walker, 2017)
      \item Applying unconditional quantile regression(Firpo, Fortin, and Lemieux, 2009), they can discuss the impact of the Clean Air Act on diffrent empirical moments of the nationwide pollution distribution
    \end{itemize}
  \end{enumerate}
\end{frame}

\section{Data}
\frame{\sectionpage}
\begin{frame}{Background and Difficulties}
  \begin{itemize}
    \item Spatially-continuous satellite measurements of pollution correlates
    \begin{itemize}
      \item "out-of-sample" predictions: build a predictive model of a pollutant of interest by correlating EPA-monitor data with the observable characteristics (van Donkelaar, Martin, Brauer et al, 2016)
    \end{itemize}
    \item This paper uses a 1km by 1km resolution daily PM2.5 concentration data of 2000-2015 (Di, Kloog, Koutrakis et al., 2016).
    \begin{itemize}
      \item Satellite measurements are biased downward for high PM2.5 levels.
    \end{itemize}
  \end{itemize}
\end{frame}

\begin{frame}{Data Construction}
  \begin{itemize}
    \item Individual-level data with pollution and racial identities
    \begin{itemize}
      \item 2000 Census long from
      \item 2001-2015 American Community Surveys
    \end{itemize}
    \item Primary comparisons focus on the non-Hispanic White and African American populations.
    \begin{itemize}
      \item These are the largest and most-documented gaps
      \item Lieber, Porter, Fernandez et al., 2017: Hispanic identity is more fluid over time than White or black racial identities.
    \end{itemize}
  \end{itemize}
\end{frame}

\begin{frame}{}
  \begin{figure}
    \centering
    \includegraphics[scale = 1]{F1.png}
    % \includegraphics[scale = .5]{F2.png}
  \end{figure}
\end{frame}

\begin{frame}{Racial Gaps in Pollution Exposure}
  \begin{itemize}
    \item The observed racial gap in mean pollution exposure has declined by \SI[per-mode=symbol]{1.0}{\micro \gram \per \cubic \meter}  in 15 years.
    \item This improvement in the Black-White pollution gap could potentially explain 4\% of the mortality gap improvement.
    \begin{itemize}
      \item Life expectancy is reduced by .61 years for each \SI[per-mode=symbol]{10}{\micro \gram \per \cubic \meter} (Pope III, Ezzati, and Dockery, 2015)
      \item Over 2000-2015, the Black-White gap in life expectancy fell from about 5 years to 3.5 years (Arias, Xu, and Kochanek, 2019).
    \end{itemize}
    \item The gap in exposure is explained by census-tract differences (about \SI{5}{\square \kilo \meter}).
  \end{itemize}
\end{frame}

\begin{frame}{}
  \begin{figure}
    \centering
    % \includegraphics[scale = .1]{F1.png}
    \includegraphics[scale = .7]{F2.png}
  \end{figure}
\end{frame}

\section{Decomposing Differences in Pollution Exposure}
\frame{\sectionpage}
\begin{frame}{}
  \begin{figure}
    \centering
    \includegraphics[scale = .4]{TB2.png}
  \end{figure}
\end{frame}

\begin{frame}{Conditional versus Unconditional Differences in Pollution Exposure}
  \begin{itemize}
    \item Differences in exposure conditional on the differences in individual characteristics.
    \item Linear Regression: for individual $i$,
    \begin{align*}
      P_i = \gamma \mathbb{1}[\text{African American}_i] + X' \beta + \epsilon_i
    \end{align*}
    \begin{itemize}
      \item $X_i$: individual income, age, education, number of children, gender, and an indicator for homeownership.
      \item weighted by survey weights
      \item SEs are clustered by commuting zone
    \end{itemize}
  \end{itemize}
\end{frame}

\begin{frame}{}
  \begin{figure}
    \centering
    \includegraphics[scale = 1]{F3AB.png}
  \end{figure}
  \begin{itemize}
    \item Individual characteristics seems to explain almost none of the differences.
  \end{itemize}
\end{frame}

\begin{frame}{Oaxaca-Blinder decinoisution}
  \begin{itemize}
    \item Formally decomposing cross-sectional differences (Oaxaca, 1973; Blinder, 1973).
    \item Observable differences in individual and household characteristics are able to explain at most 8 percent of the gap in mean differences in any given year.
  \end{itemize}
\end{frame}

\begin{frame}{}
  \begin{figure}
    \centering
    \includegraphics[scale = .6]{TB4.png}
  \end{figure}
\end{frame}

\begin{frame}{Differences at Different Quantiles of the Pollution Distribution}
  \begin{itemize}
    \item Individual or household characteristics are able to explain differences in pollution exposure at other parts of the pollution distribution.
    \item DiNardo, Fortin, and Lemieux (1996): re-weighted kernel density estimate
    \begin{itemize}
      \item estimate what the entire distribution of African American pollution exposure would look like if African Americans had the same observable characteristics
    \end{itemize}
    \item Again, individual characteristics are able to explain little of the observed pollution gap throughout the distribution.
  \end{itemize}
\end{frame}

\begin{frame}{}
  \begin{figure}
    \centering
    \includegraphics[scale = .6]{FB2.png}
  \end{figure}
\end{frame}

\begin{frame}{Controlling for Naeghborhood Characteristics}
  \begin{itemize}
    \item Socioeconomic Characteristics
    \begin{itemize}
      \item African Americans tend to be concentrated in census tracts with relatively disadvantaged neighbors.
    \end{itemize}
    \begin{figure}
      \centering
      \includegraphics[scale = .45]{F3C.png}
    \end{figure}
  \end{itemize}
\end{frame}

\section{The Clean Air Act and Relative Changes in Pollution Exposure}
\frame{\sectionpage}
\begin{frame}{}
  
\end{frame}

\section{Conclusion}
\frame{\sectionpage}
\begin{frame}{}
  
\end{frame}

\end{document}
