\documentclass[dvipdfmx,11pt]{beamer}
%\graphicspath{{fig_tab/os_presentation/20201215/}}
\usepackage{lipsum}
\usetheme{verona}
\usepackage{bxdpx-beamer}
\usepackage{pxjahyper}
\usepackage{minijs}
\usepackage{mathpazo}
\usepackage{amsmath,amssymb}
\usepackage{graphicx}
\usepackage{array}
\usepackage{tikz}
\usepackage{wrapfig}
\usepackage{float}
\usepackage{here}
\usepackage{lscape}
\usepackage{ascmac}
\renewcommand{\kanjifamilydefault}{\gtdefault}
\hypersetup{% hyperrefオプションリスト
 setpagesize=false,
 bookmarksnumbered=true,%
 bookmarksopen=true,%
 colorlinks=true,%
 linkcolor=blue,
 citecolor=blue,
 urlcolor = magenta
}
\setbeamertemplate{navigation symbols}{}

\title[Dellavigna and Linos (2022, Econometrica)]{RCTs to Scale: Comprehensive Evidence from Two Nudge Units}
\subtitle{Dellavigna and Linos (2022)}
\author{Reviewed by Reio TANJI}
\date{Apr. 12th, 2022 \\ Ohtake-Sasaki Seminar}
\institute[OU Graduate School of Economics]{Osaka University, Graduate School of Economics}

\begin{document}

\begin{frame}\frametitle{}
\titlepage
\end{frame}

\begin{frame}{Abstract}
  \begin{itemize}
    \item A meta-analysis of Nudge interventions.
    \begin{itemize}
      \item A unique dadtaset that assembles 126 RCTs covering 23 million individuals (two of the largest Nudge Units in the U.S.).
    \end{itemize}
    \item Comparing these samples found a difference in the size of average impacts.
    \begin{itemize}
      \item Evidence from academic journals shows very large and significant impact, while that from Nudge Units are smaller.
    \end{itemize}
    \item Three dimensions accounts for these defferences.
    \begin{enumerate}
      \item Statistical power of trials
      \item Characteristics of the interventions
      \item Selective publication
    \end{enumerate}
    \item Among them, selective publication explains about 70\% of the difference in effect sizes.
  \end{itemize}
\end{frame}

\end{document}
