\documentclass[dvipdfmx,11pt]{beamer}
%\graphicspath{{fig_tab/os_presentation/20201215/}}
\usepackage{lipsum}
\usetheme{verona}
\usepackage{bxdpx-beamer}
\usepackage{pxjahyper}
\usepackage{minijs}
\usepackage{mathpazo}
\usepackage{amsmath,amssymb}
\usepackage{graphicx}
\usepackage{array}
\usepackage{tikz}
\usepackage{wrapfig}
\usepackage{float}
\usepackage{here}
\usepackage{lscape}
\usepackage{ascmac}
\renewcommand{\kanjifamilydefault}{\gtdefault}
\hypersetup{% hyperrefオプションリスト
 setpagesize=false,
 bookmarksnumbered=true,%
 bookmarksopen=true,%
 colorlinks=true,%
 linkcolor=blue,
 citecolor=blue,
 urlcolor = magenta
}
\setbeamertemplate{navigation symbols}{}

\title[Dellavigna and Linos (2022, Econometrica)]{RCTs to Scale: Comprehensive Evidence from Two Nudge Units}
\subtitle{Dellavigna and Linos (2022)}
\author{Reviewed by Reio TANJI}
\date{Apr. 12th, 2022 \\ Ohtake-Sasaki Seminar}
\institute[]{Osaka University, Graduate School of Economics}

\begin{document}

\begin{frame}\frametitle{}
\titlepage
\end{frame}

\begin{frame}{Abstract}
  \begin{itemize}
    \item A meta-analysis of Nudge interventions.
    \begin{itemize}
      \item A unique dadtaset that assembles 126 RCTs covering 23 million individuals (two of the largest Nudge Units in the U.S.).
    \end{itemize}
    \item Comparing these samples found a difference in the size of average impacts.
    \begin{itemize}
      \item Evidence from academic journals shows very large and significant impact, while that from Nudge Units are smaller.
    \end{itemize}
    \item Three dimensions accounts for these defferences.
    \begin{enumerate}
      \item Statistical power of trials
      \item Characteristics of the interventions
      \item Selective publication
    \end{enumerate}
    \item Among them, selective publication explains about 70\% of the difference in effect sizes.
  \end{itemize}
\end{frame}

\section{Introduction}
\frame{\sectionpage}

\begin{frame}{Nudge Interventions}
  \begin{itemize}
    \item Nudge 
    \begin{itemize}
      \item "\textit{choice architecture that alters perple's behavior in a predictable way without forbidding any prtions or significantly changing their economic incentives.}"
      \item have become common in the literature in fields such as economics, political science, public health, decision-making, and marketing.
    \end{itemize}
    \item Nudge Units: larger-scale applications by governments.
    \begin{itemize}
      \item Behavioral science to improve government services.
      \begin{itemize}
        \item ideas42 in the U.S. (2008)
        \item the UK's Behavioural Insights. (2010)
        \item Office of Evaluation Sciences (2015)
      \end{itemize}
      \item As of last count, there are more than 200 Nudge Units globally.
    \end{itemize}
  \end{itemize}
\end{frame}

\begin{frame}{What this paper did}
  \begin{itemize}
    \item A meta-analysis which colloborates with two major Nudge Units
    \begin{itemize}
      \item BIT North America: conducts projects with multiple U.S. local governments 
      \item OES: collaborates with multiple U.S. Federal agencies.
    \end{itemize}
    \item They conducted a total of 165 trials testing 347 nudge treatments, affecting almost 37 million participants.
    \item This paper avails 126 RCT trials, involving 241 nudges and collectively impacting over 23 million participants.
  \end{itemize}
\end{frame}

\begin{frame}{Literature and Contribution}
  \begin{itemize}
    \item Trials to nudges: Benartzi et al. (2017) and Hummel and Maedche (2019) summarizes over 100 published nudge RCTs.
    \item However, most of them are not have been documented in working papers or academic publications.
    \begin{itemize}
      \item BIT and OES conducted 165 trials, but 87\% of them are not published as papers.
    \end{itemize}
    \item Evidence from their unique data set differs from a traditional meta-data analysis in:
    \begin{enumerate}
      \item The large majority of have not previously appeared in academic journals.
      \item No scope for selective publications.
    \end{enumerate}
  \end{itemize}
\end{frame}

\begin{frame}{Summary of Results}
  \begin{itemize}
    \item In the 26 papers in the Academic Journals sample, the average impact of nudge interventions raised take-up rate by 8.7 percentage points (33.4\%).
    \item Including all 126 trials by Nudge Units showed an unweighted impact of 1.4 percentage point (17.3\%).
    \begin{itemize}
      \item The impace is highly statistically significant, but there is large difference between two samples. 
    \end{itemize}
    \item 
  \end{itemize}
\end{frame}

\section{Setting and Data}
\frame{\sectionpage}

\section{Impact of Nudges}
\frame{\sectionpage}

\section{Nudge Units Versus Academic Journal Nudlges}
\frame{\sectionpage}

\section{Introduction}
\frame{\sectionpage}

\section{Discussion and Conclusion}
\frame{\sectionpage}

\end{document}
