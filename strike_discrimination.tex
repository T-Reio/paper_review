\documentclass[dvipdfmx, 12pt]{jsarticle}
\usepackage{newtxtext,newtxmath}
\usepackage{mathpazo}
\usepackage{amsmath,amssymb}
\usepackage{array}
\usepackage[hiresbb]{graphicx}
\usepackage{tikz}
\usepackage{textcomp}
\usepackage{dcolumn}
\usepackage{caption}
\usepackage{siunitx}
\usepackage{colortbl}
\usepackage{multirow}
\usepackage{hhline}
\usepackage{calc}
\usepackage{tabularx}
\usepackage{threeparttable}
\usepackage{graphicx}
\usepackage{wrapfig}
\usepackage{hyperref}
\hypersetup{
    colorlinks = true,
    citecolor = blue,
    linkcolor = magenta,
    urlcolor = blue,
}
\usepackage[authoryear]{natbib}
\bibpunct[:]{(}{)}{;}{a}{}{,}
\bibliographystyle{abbrvnat} %dinat plainnat rusnat ksfh_nat unsrtnat humannat

\usepackage[top=25truemm,bottom=30truemm,left=25truemm,right=25truemm]{geometry}

\renewcommand{\contentsname}{Contents}
\renewcommand{\figurename}{Figure }
\renewcommand{\tablename}{Table }
\renewcommand{\refname}{References}

\begin{document}

\title{Discrimination in MLB's Pitch-Call: \\
The Effect on the Observed Performance and the Salaries}
\author{丹治 伶峰 Reio Tanji \\
大阪大学大学院 経済学研究科 博士後期課程 \\
Osaka University, Graduate School of Economics}
\date{}
\maketitle

\section*{Abstract}

\leftskip = 25pt
\rightskip = 25pt

This paper identifies whether or not discrimination exists in the professional baseball league in the United States. We consider three agents in this setting: players (workers), umpires (those who evaluate the workers' outcome), and team managers (those who offer players contracts). Using huge pitch-by-pitch tracking data of the Major League Baseball reveals that umpires from North America favor players with the same region: call more strikes when a pitcher from North America and a batter from other regions are facing. The impact of this is dramatic: players from other regions lose their chances to hit by the unfair pitch call, which values to loss of about \$130,000 for in three years.

\leftskip = 0pt
\rightskip = 0pt

\medskip

\noindent
\small

\paragraph{JEL Classification:}D91, J01
\paragraph{keywords:}sports, discrimination, in-group bias, baseball

\noindent

\normalsize

\section{Introduction}

Studies in labor economics have dealt with discrimination for a long time. Favorism of individuals who belong to the same group as the evaluator is called in-group bias, which have serious implications in the labor market.

This paper presents an empirical study on in-group bias using a large dataset from Major League Baseball (MLB). In this study, we use a radar-based high-precision ballistic measurement device called \textit{Trackman}. This has been installed in all franchise ballparks of MLB teams from 2015 to 2019 to record and disclose detailed information on the trajectory of balls thrown by pitchers and the launch angle and launch velocity of balls hit by batters. The particular focus of this paper is on the location of the pitch as it passes home plate. When a pitch is thrown by a pitcher and the batter does not hit it (takes), the plate umpire judges whether it is a strike or ball. Although there are strict rules governing where a pitch should be called a strike, the actual decision is based on the umpire's visual inspection, and it is common that two pitches that pass through the exact same location are given different calls, especially near the edge of the strike zone. This paper considers this system can identify the objective information of the pitches. Comparing this to the actual pitch calls, an evaluation by the plate umpire, we can examine how their demographic characteristics affect the evaluator's (plate umpire) evaluation of the achievement of the workers (players).

Because we can use workers' (athletes') performance as an objective measure, and because it constitutes a large dataset, there has been a large body of literature on discrimination using sports data. Among them, the most relevant to this paper is \citet{Parsons_etal_2011}. They use the MLB's tracking data to analyze the impact of racial discrimination by the plate umpire, as in this paper. The most important advantage of this dataset is the random assignment of umpires and almost external variation in installations of "\textit{QuesTec}."\footnote{"\textit{QuesTec}" is a technology installed in some of the franchise stadiums of the MLB teams, to provide feedback to umpires on the accuracy of their decisions.} They found that racial differences in pitchers and umpires increased the rate of strike calls. \citet{Tainsky_etal_2015} also conducted a follow-up study using the same data.

In comparison to these previous studies, we emphasize the following new contributions in this paper. First, this paper focuses on how umpires' bias distorts the performance stats of players and quantifies the magnitude of the effect on the amount of salary a player receives. Objective information about pitch locations enables us to see how likely a specified pitch should be called strike and every actual call. It allows us to argue the size of the impact that umpires exhibit through pitch calls. What is vital is this bias may distort the performance stats of discriminated players appear downward from what they should be. Team managers offer contracts based on the observed performance indicators. It may cause players discriminated against by umpires to be forced to sign contracts than those favored. This paper avails advantage of the regulation of a video-based judgment review system called "Manager Challenge." Using this system, we can capture almost all of the indirect negative impact of umpire bias on the players' compensation. This study is unique because it deals with the two resources of discrimination in the same context. One is by the umpires(the evaluators in the workplace), and the other is by the managers (the principals) who offer contracts to the players. In a typical workplace, it is the middle managers who evaluate the performance of workers. Then, based on their reports, the heads of the company decide whether to renew the contract with the workers. This environment is similar to the example we will address in this paper, and provides suggestive results for distinguishing between discrimination by middle managers and discrimination by the president. In this viewpoint, this paper also has implications for a more general context of labor economics.

Second, this paper provides a more detailed discussion of the endogeneity problem implied by pitch tracking data, as is argued in the prior literature. For example, pitchers discriminated against may avoid pitching to the edge of the strike zone. It is also likely that batters will swing more aggressively, and we cannot observe such pitches as ones to be called. If so, biased calls may result in worse performance in the batted balls hit by the minority players. This paper explores this concern by analyzing batters' behavioral responses toward pitches.

The following is a summary of the results. First, umpires from the U.S. were found to make pitch calls influenced by the nationality bias based on the batter's nationality. Specifically, in situations where the pitcher is from the same country as the umpire and the batter is not, the umpire calls strikes at a higher rate, i.e., makes a decision in favor of the pitcher. Based on the average probability of a strike call estimated from the information about the two-dimensional passing position of the pitch, the variance of the probability in such a situation is about 0.3 percentage points per taken pitch.

The second result attempts to convert this into a monetary value. Based on the first result, minority batters who consistently have their names in the lineup are called about 38 more strikes per year. It worsens the performance stats that the managers observe, resulting in a loss of approximately \$40,000 in compensation per year that they could have received.

The third result concerns the issue of endogeneity in players' decision-making. Discrimination by plate umpires may cause behavioral changes in players who perceive it, but the analysis conducted in this paper did not provide strong evidence to support these behavioral changes.

The rest of this paper proceeds as follows. In Section 2, we review the literature on discrimination, especially on empirical literature in the context of sports. Section 3 explains the institutional background of baseball or Major League Baseball, and the decision making that plate umpires do in pitch-calls. Section 4 is provides on explanation about the empirical data. Section 5 shows the empirical model in the identification, and Section 6 provides the results. Section 7 further discusses the results, and in Section 8, we conclude the study.

\section{Literature Review}

Because of the availability of rich datasets, much empirical literature has turned MLB into studies on in-group or racial bias. One of the oldest and well-known literature of them is \citet{Scully_1974}. They estimated the profit function of the MLB teams and analyzed the relationship between the characteristics of the players and the team's revenue proxied by the number of attendance. They pointed out that an increase in the percentage of African American players on the team's roster harms the team's revenue and discussed consumer bias by the audience. A similar study took place by \citet{Hill1982}, who showed that if the home team's starting pitcher announced in advance is a minority pitcher, the attendance for that game decreases. \citet{Nardinelli1990} looked at the MLB player card market and found that non-white pitchers with comparable performance priced about 13\% lower than white pitchers (10\% lower for non-pitchers).

Another target of research on discrimination using data from professional sports is discrimination by the teams that offer contracts with players.  We can observe the quantified contribution of each player to the team through the performance stats like the number of home runs or strikeouts.  We can also see the amount of compensation for the players, in other words, the evaluation by the team managers. This situation allows us to argue whether the salary for a player is adequate or not. Many studies test the existence of racial discrimination by team managers. \citet{Krautmann2000} regarded performance stats as the performance indicators and showed that non-white players are more cost-effective than white players. That is, teams can acquire non-white players with cheaper contracts. Non-white players are discriminated against by companies (teams) in the player labor market. Similarly, \citet{Holmes2011} conducted analyses on discrimination by estimating the relationship between players' performance indicators and compensation.

\section{Background}

\subsection{Pitch Call in the MLB}

First, we will describe the details of judging balls and strikes. Hereafter, we call it "pitch calls."

MLB umpires usually operate with four umpires. Each is responsible for plate umpire and the respective base umpires at the first, second, and third bases. The four umpires make up a unit called a "crew,"  As a rule, the MLB assigns one crew to each game in the regular season. Each umpire does not specialize in any single position but rotates all four ones in order. Thus, an umpire will be a plate umpire in about one out of every four games. Many umpires are graduates of technical schools and, like the players, work their way up through the ranks of the lower divisions, called Minor League Baseball (MiLB), to earn MLB contracts. The participation fees paid to umpires are comparable to those to highly skilled professionals\footnote{For more detailed information about MLB umpires, see \citet{umpire_detail}, for example.}, and since their decisions are televised throughout the U.S. and around the world, they have an incentive to make more accurate and sincere decisions.

One of the most important tasks of the plate umpire in a baseball game is to make pitch calls. If the batter takes (does not swing) the pitch, the plate umpire calls "ball" or "strike" based on the location of the pitch as it crosses home plate. The strike zone is defined as follows:

\begin{quote}
  The official strike zone is the area over home plate from the midpoint between a batter's shoulders and the top of the uniform pants -- when the batter is in his stance and prepared to swing at a pitched ball -- and a point just below the kneecap. In order to get a strike call, part of the ball must cross over part of home plate while in the aforementioned area.
  \begin{flushright}
    quoted from \citet{official_strike_zone}
  \end{flushright}
\end{quote}

In each plate-appearance, if the umpire calls the third strike (caught by the catcher) before the batter hits the pitch into the fair territory, he is out (strikeout). On the other hand, if four pitches pass outside the strike zone and the batter takes them, he is entitled the first base. The number of balls and strikes during the at-bat (pitch count) also has a significant effect on the result of each plate-appearance (see Table \ref{byCounts_stats}), and whether an individual pitch is called a strike or a ball is a very pivotal issue in the players' performance. (see Table \ref{byCounts_stats}).

\begin{table}[ht]
    \centering
    \caption{wOBA by Pitch Counts}
    \label{byCounts_stats}
    \scalebox{.9}{
    \begin{threeparttable}
      \begin{tabular}{ccccc} 
        \hline
        Balls $\backslash$ Strikes & 0 & 1 & 2 &  \\ \hline
        0 & .406 & .372 & .173 \\
        1 & .413 & .390 & .186 \\
        2 & .430 & .407 & .210 \\
        3 & .678 & .580 & .389 \\ \hline
      \end{tabular}
      \begin{tablenotes}
        \footnotesize
        \item wOBA (the abbreviation for Weighted On-Base Average) is one of the most important and popular statistics to evaluate batters. See \url{https://library.fangraphs.com/offense/woba/} for the details.
      \end{tablenotes}
    \end{threeparttable}
    }
\end{table}

Although the strike zone is well defined, not all pitches are judged as specified above. Figure \ref{pitchcall_illustration} illustrates the physical pitch location and the actual decision of a pitch to a batter. The bold rectangle in the figure shows the strike zone according to the definition, and a pitch that passes inside this zone is the one that should be called a strike. In practice, however, it can be observed that a pitch that passes through the strike zone can be called a ball, and vice versa. Each pitch is judged visually by the plate umpire, and as we will see later, this judgment cannot be overturned once it has been called. Therefore, if the umpire makes a discriminatory decision (regardless of whether the tendency occurs consciously or not), the effect will be to the detriment of minorities in the form of a lower performance index due to an unfavorable decision.

\begin{figure}[ht]
    \caption{Pitch location and Pitch calls}
    \label{pitchcall_illustration}
    \centering
    \includegraphics[scale = 1]{fig_tab/trout_zone.png}
\end{figure}

\subsection{Discrimination in Major League Baseball}

So-called modern \textit{Major League Baseball} (MLB), one of the biggest sports league in the United States (or in the World), was established in 1876. Like the rest of American culture, its history has been a history of struggle against racism.

In 1867, \textit{Association of Amateur Base Ball Players} rejected African American menbership. Then, in 1876, owners of the professional \textit{National League} adopted a "gentleman's agreement" to keep black players out. The excluded black players founded their own leagues, \textit{Negro League}, and had formed a unique culture in response to their white counterparts. It took more than 70 years to see the first African-American player, Jackie Robinson, in the MLB in 1947.

After his debut, more and more teams signed black and foreign players, and now about 28\%\footnote{The rate of foreign players that entered at least one game of MLB in the seasons from 2015 to 2019.} of all players are from overseas.  In addition, in 2020, the MLB announced that the Negro League will be recognized as a major league and that individual and team records set by black players will be treated as official.

However, these efforts are also symbolic in that international and black athletes have suffered from various forms of discrimination to date. While the channels of discrimination include statistical discrimination based on false stereotypes, there are also scattered examples of stereotypes that seem to be taste-based, such as minority players are disgusted by discriminatory comments or players being punished for behaving in a manner that disparages minorities in interviews or on social networking sites. For example, in the 2017 World Series, Yu Darvish, a Japanese pitcher, faced a discriminatory gesture by his opponent during the game, for which he later expressed his apology.

These situations are alike in the modern society of the United States, which is still facing various incidents of discrimination, and understanding it in the MLB is significant in visualizing the existence of the problem of discrimination in society as a whole.

\subsection{Pitch Tracking System in the MLB}

Then we explain the institutional background of how the MLB acquires pitch tracking data in each game and how they are utilized in empirical analyses.

Tracking data includes physical information about players and balls using cameras and radar technology. The system accumulates the data pitch-by-pitch and contains the results of each play and tag about players and games. In MLB, the speed and rotation of pitches thrown by pitchers and the two-dimensional position of the ball as it passes home plate are recorded for each pitch (called pitch location), along with other information about the game. Through each game, information about 300 pitches is accumulated. Since the MLB regular season consists of about 2,400 games\footnote{The regular-season game is held to determine the champions of the six divisions (2 leagues $\times$ 3 divisions). Each team plays 162 games in total and according to the results of them, the right to enter the postseason games to compete the whole-league champion, called World champion, is determined.}, a total of about 720,000 pitches of data are obtained in every single season.

The first MLB team that introduced equipment to acquire tracking data was in 2004. The system, called \textit{Pitch F/X}, is a technology that uses cameras to capture the trajectory of each pitch and the movement produced by the rotation of the ball, which generates the Magnus effect, for example. Initially, the system was a tool for teams to give objective feedback to their players, not for providing data for the analysts or fans outside the organizations. Thus, until 2008, some MLB franchise ballparks had installed the system while others have not. Since this dataset serves objective information on pitch locations, they started using this to provide feedback on the pitch call by plate umpires to improve the quality of calls. In \citet{Parsons_etal_2011}, the authors exploited the exogenous differences in the presence of this feedback system, called \textit{QuesTec}, among ballparks to analyze changes in pitch calls as a proxy variable for the degree of monitoring\footnote{In the ballparks that are equipped with \textit{QuesTec}, every pitch call is to be recorded and evaluated by the League. Those without the system, on the other hand, do not observe the actual locations of the pitches, so umpires could feel less monitored, which may lead to more biased judges.}. Then in 2008, all of the 30 MLB teams installed \textit{Pitch F/X} system in their franchise ballparks. Since then, the dataset has been published via \citet{baseballsavant}, and everyone can download it for free.

In 2015, \textit{Pitch F/X} was replaced by another tracking system, \textit{TrackMan}\footnote{\textit{Pitch F/X} system continued working until 2020 season, but the official version of the dataset is tracked via \textit{TrackMan}.}. This new instrument uses Doppler radar to capture balls instead of hi-speed cameras. It made it possible to capture the exit velocity or exit angle of the batted balls in addition to pitches. This system allows us to estimate the quality of the batted balls more objectively. Based on the physical information about the ball, we can quantify the probability that the ball will be a hit or a home run. Using the findings of sabermetrics we can also calculate how many runs the batted ball creates. All the MLB ballparks were installed \textit{TrackMan} at the start of the 2015 season, so data is available for almost all the games played in each year\footnote{Most of the official MLB games takes place at MLB franchise ballparks. The exceptions are games outside the U.S. (such as in Japan, Mexico, or China...),  but the number of such games is few. Portable devices enabled to acquire data even in stadiums without \textit{TrackMan}, as was the case in the United Kingdom in 2019.}. Tagged with the information of the games and players, The data tracked by the radar constitutes a vast dataset. It includes more detailed information such as defensive shifts recorded by high-precision cameras and the efficiency of the routes used by fielders to track batted balls\footnote{Some of the information on the fielders' defense is not available to the public, but we can observe aggregate data on television, through Internet coverage, or data sites such as \citet{baseballsavant}.}. This system is called \citet{statcast}.

\subsection{Video-Based Judge Review System}

Another essential institutional background for the analysis in this paper is the introduction of the replay system. At the start of each game in the regular season, the managers of both teams get a one-time right to request a replay review of a close play. If they exercise the right in the short interval before the next play\footnote{30 seconds limitation was implemented in 2017, and the timeframe was shortened to 20 seconds in 2020.} (pitch, throw, etc.) occurs, the umpires make it reviewed according to a prescribed procedure. The umpires cannot refuse the claim. If the judge is overturned, i.e., the coach's request is successful, the number of times entitled will not decrease, and the manager can request another video judgment if necessary. In addition, the umpires may request a review, which does not take away the right to challenge given to the managers of either team. Nowadays, each team has its staff outside the dugout to review the video, and they are responsible for determining whether or not the play should be challenged and informing the coach.

The review requested by the manager is called  "Manager Challenge," and the one by the umpire is called "Umpire Review." In all cases, it is not the umpires in charge of each game that reviews the judge but another referee deployed at a specialized facility called the "Replay Command Center" in New York. Of course, those who judge these reviews are official umpires of the MLB. From the center, they can check all the camera footage of each game. They scrutinize the replay without the initial judgments, so there is no need to worry that the umpires' initial decision is influenced by their desire to protect their dignity, for example.

Judge Review, which had been introduced in 2008 only for judging home runs, was first piloted in 2014 in a method that also gave managers the right to request it, and the following year, in 2015, the current regulation was put into operation. The managers can challenge almost all plays (fair or foul, in or out of the ballpark, fan interference) except for pitch calls. This expansion allows us to quantify, in the context of discrimination research, most of the effects of discrimination suffered by minority players.

Figure \ref{fig_umpire_review} illustrates the plays in which discrimination by umpires can occur before and after the introduction of Judge Review. Before the introduction of this system, almost all the plays other than those of homeruns, once an umpire declares the decision, no one could overturn it, and minority players were disadvantaged if the judge were biased. Since the implementation of the replay verification system, such discriminatory decisions will be more likely to be overturned upon request from the coach. On the other hand, concerning pitch calls, as explained earlier, the impact of discrimination continued to exist: umpires' decisions are the final ones. Thus, the most vital channel of umpire discrimination within MLB grounds after 2015 is the judgment call for ball-strike decisions, which is the subject of this paper's analysis\footnote{Two other possible channels are the half-swing, where the batter stops swinging midway, and the batted ball in front of the ball and base umpire's position, where the play is relatively easy to make a good decision. Section 6 also discusses the magnitude of the impact of considering these plays.}.

\begin{figure}[t]
  \centering
  \caption{Illustration of the umpire review}
  \includegraphics[scale = .35]{fig_tab/umpire_discrimination.png}
  \label{fig_umpire_review}
\end{figure}

\subsection{What is Unique to See This Context?}

Figure \ref{fig_path_disc} draws the pathways of the effects of discrimination that occur against minority athletes. Prejudice in pitch calls leads to disadvantages for minorities through two paths: direct and indirect effects.

The direct impact means that an umpire's decision hinders a player's ability to perform better at-bat or in the game. As is mentioned in previous sections, having a strike called instead of a ball has a significant negative impact on the batter's subsequent performance in the at-bat (if it is the third strike, the batter is out and is not allowed to stay in the at-bat any longer). It makes it impossible for minority players to perform as well as they would have without discrimination.

On the other hand, the indirect impact corresponds to the arrow extending from "Biased performance stats" to "Biased contract" in Figure \ref{fig_path_disc}. The indirect effect occurs when the team management to which the players belong offers (or breaks) the contract for the following year based on performance stats. Managers evaluate the minority players according to their performance stats that is inferior to what it should be. Since managers are unaware of the observed downward bias in performance metrics, or even if they are aware of it, they do not need to take care of it and do not offer appropriate contracts. As a result, even if the team's management does not discriminate against minorities\footnote{Whether they are caused by their taste-based discrimination or consumer discrimination is not the issue here.}, it indirectly harms the players by accepting biased judgments by umpires. 

\begin{figure}[t]
  \centering
  \caption{Path of the Effects of Discrimination against Minority Players}
  \includegraphics[scale = .35]{fig_tab/path_of_disc.png}
  \label{fig_path_disc}
\end{figure}

This paper aims to distinguish the three types of discrimination (direct/indirect effects of biased calls, and the discrimination in player evaluation by team management) in Figure \ref{fig_path_disc} and discuss which sources we should pay more attention to improving the situation.

As mentioned in the previous section, almost all discrimination in the playfield occurs in pitch calls. Therefore, estimating this allows us to quantify the conclusive impact of discrimination. To the best of the author's knowledge, this paper is the first study to tackle this question, where was also mentioned in \citet{Parsons_etal_2011}

\section{Data}

This section then describes details of the data in the analysis. To create the necessary variables, we need (1) tracking data for each pitch, including pitch location, (2) data on the country of origin of players and umpires, and (3) data on players' performance stats and contracts. In the primitive analysis, (1) and (2) are merged to create a unique pitch-by-pitch dataset. 

The analysis will cover the seasons 2016-2018. Data on games and plays cover the regular season. In addition, since the managers design the players' contracts based on the previous year's performance, the information on player participation compensation described in Section 4.3. is tagged with a one-year lag from the respective year (annual salary for 2017-2019 is merged to the stats for 2016-2018)\footnote{In 2020, due to the COVID-19 outbreak, the number of regular-season games was changed from 162 to 60, and the player's earnest money was limited to 37 percent of the contract they originally signed (see \citet{salary_cut}). In addition, games of the MiLB, the MLB's subordinate organization, were cancelled, and each player was given the right to decide whether or not to participate in the season due to health risks, and a number of players actually withdrew from the games themselves.
For these reasons, the study excluded 2019 and 2020 data from the sample.}.

\subsection{Pitch Tracking Data}

First, we start to explain pitch tracking data.
\citet{baseballsavant} aggregates tracking data acquired during MLB games and publishes quantitative analysis that predicts players' performance using physical information such as batted ball velocity and qualitative characteristics of pitches. Much of the data used in the study is available to the public, and we can visit the site to get pitch-by-pitch datasets tied to game and player information.

The total number of games played from 2016 to 2018 was 9,722 and the number of pitches thrown during the games was 20,189,706. In our main analysis, we will focus on the pitches that batters forfeited without swinging at ($N = 1,130,048$).

\begin{table}[ht]
  \caption{\# of pitches with potential bias}
  \centering
  \label{pitches}
  \begin{tabular}{lcc}\hline
    Umpire & North America & Other\\ \hline
    Umpire, Batter, and Pitcher from the same region & 538,117 & 5,266\\
    Only the umpire from other region & 80,260 & 44,520 \\
    Possible bias to the batter & 266,357 & 11,930 \\
    Possible bias to the pitcher & 163,313 & 20,285 \\ \hline
  \end{tabular}
\end{table}

Table \ref{pitches} lists the number of pitches  devided by plate umpires, pitchers, and the region the batter is from. As discussed below, the majority of MLB certified umpires are from the U.S., so the ratio is somewhat skewed, but in situations where either the pitcher or the batter is from the same country with the place umpire, i.e., where plate umpire discrimination is likely to occur, the number of pitches thrown should be at least 10,000. A sample of about 10,000 pitches can be obtained.

\subsection{Nationalities of the Players and Umpires}

\input{fig_tab/summary_players.tex}
\input{fig_tab/players_list.tex}

Table \ref{player_umpire_summary} and \ref{player_country} display summaries of the birthplaces the players and the umpires are from. During the sample period, 2475 players and 108 umpires are rostered as the MLB players and umpires, respectively. The vast majority about both the players and the umpires is the United States, especially for the umpires. Umpires from the United States account for almost 90\% of all the umpires.

\citet{chadwick} contains almost all the players and officials involved in modern American sports. The published dataset includes the year in which a player or an umpire first/last played an MLB game, as well as the unique ID of each player on major websites such as \citet{baseballsavant}, \citet{bbref}, and \citet{fg}. Using the data, the author can merge the tracking data described above with information on the country of origin of players and umpires scraped from \citet{bbref}, and data from \citet{fg} as described below.

One of the most important determinants of the strike zone is the height of the batter. If the batter's height varies significantly depending on the player's region of origin, these characteristics may be the source of differences based on race. However, we did not identify any statistically significant differences in mean height between regions in our sample. This may be due to a survival bias, in which a certain level of height is required for a player to be able to compete in MLB games, but in any case, it explains why there is not a significant difference in the physique of the players that the umpires actually face.

\input{fig_tab/pitcher_stats.txt}
\input{fig_tab/batter_stats.txt}

\subsection{Contracts}

The data for player contracts and the performance are obtained from \citet{bbref} and \citet{fg}. These are privately owned and operated sites that calculate and publish objective player evaluation indicators based on sabermetrics. 

Only a few players are able to play under a season-long contract in MLB, and many are temporarily promoted to MLB and then fired or demoted to MiLB and re-signed. As a result, the number of players for whom we have data on salary during the period in question is small compared to the number of players who have played at least one game. In this section, we will discuss the results of our analysis. The sample size was 611.

\input{fig_tab/sal_sum.tex}

\section{Identification Strategy}

This section describes the empirical strategy to identify the existance and the size of the effect of umpires' discrimination. Using the variables described in Section 4, the estimation equation using the ordinary least squares method is defined as follows.

For each individual pitch $i$ thrown by pitcher $p$, to batter $b$, and called by plate umpire $u$,

\begin{align}
  \mathbf{1}(\text{Strike}_i | \text{Called}_i) = 
  \beta_1 + \beta_2 \text{Nationality}_i + \beta_3 & \hat{f}(\text{plate\_x}_i, \text{plate\_z}_i) \notag \\
  &+ \beta_4 \mathbf{X}_{i} + \delta_b + \gamma_p + \lambda_{u} + u_i. \label{regression_main}
\end{align}

$\mathbf{1}(\text{Strike}_i | \text{Called}_i)$ is a dummy for that the pitch $i$ is called strike, given that the batter takes the pitch. The parameters of interest are $\beta_2$, which correspond to the average effects of the nationality of the pitcher, the batter, and the umpire. $\mathbf{X}_i$ stands for the pitch-, game-, or situation-specific characteristics, and $\delta$, $\gamma$, and $\lambda$ are the pitcher, batter, and umpire's fixed effects, respectively.

The nationality dummies include roughly 4 statuses in sum:
\begin{enumerate}
  \item All of the related agents (the pitcher, the batter, and the plate umpire) are from the same country (baseline),
  \item Either the pitcher or the batter is from the same country as the plate umpire while the other is from another one,
  \item The pitcher and the batter are from the same country, and the plate umpire is not,
  \item Both the pitcher and the batter are from different countries with the plate umpire.
\end{enumerate}

Theoretically, as a whole, plate umpires are likely to favor players from the same country. Therefore, discrimination by the umpire may occur in the situation (2). Each case, furthermore, is classified according to the nationality of the plate umpires. The vast majority of the official MLB umpires are from the United States,  so those from other countries are regarded as a minority. This paper distinguishes discrimination against the minority from one by minority umpires against the majority (from the United States) players.

The third term of the equation (\ref{regression_main}) is the predicted value of the average probability that the pitch is called strike, using the 2-dimensional pitch location. This probability is expressed by $\text{Pr}(\text{StrikeCalled} |\text{plate\_x}, \text{plate\_z})$, where $\text{plate\_x}$ and $\text{plate\_z}$ are the horizontal and vertical locations of each pitch. As is described in Section 3.1., It is not unusual for two pitches that pass through the exact same area to be judged differently. Thus, to estimate the effect of the umpire discrimination quantitatively, we should control the average probability of the called strikes given the pitch locations, instead of comparing the ratio of the called strikes among different nationality statuses. To control the information from the pitch location linearly, we use a generalized additive model (GAM) to estimate the density function of strike-call, conditional on the location of the pitches\footnote{This methodology is often used to quantify the catchers' technique to get more called strikes.}. The following is the equation to estimate the value of $\hat{f}$.

\begin{align}
  1\{\text{StrikeCalled}\} & \sim \text{Bin}\left( \theta = \dfrac{1}{1 + \exp(-g(\mathbf{x}))}\right) \label{logit} \\
  \text{where } g(\mathbf{x}) & = s(\text{plate\_x}) + s(\text{plate\_x}, \text{plate\_z}) + s(\text{plate\_z}) \label{spline}
\end{align}

The function $g(.)$ in equation (\ref{spline}) is a two-dimensional spline regression of the pitch locations. Taking logit transformation of $g(.)$ (equation (\ref{logit})), we can obtain the estimated probability of called strikes for each individual pitch with pitch location. Including this term into the OLS regression in equation (\ref{regression_main}), we can quantify the impact of nationality on pitch calls. If there is no plate-umpire discrimination against the minority players, no deviation is to be observed from the average probability for each location: the rules regarding pitch calls state that plate umpires are to call pitches according only to the pitch locations.

Almost all of the studies on situational strike zone using tracking data suggest that the size of the strike zone varies with pitch count and with the batter's handedness. The latter variation is due to the reversal of the batter's standing position\footnote{Figure \ref{fit_gam_02} and \ref{fit_gam_30} suggest that the strike zone is not perfectly symmetrical between right and left handed batters. This is said to be related to the fact that the catcher's catching technique (called "framing") is also likely to affect the pitch call, and that most catchers who play in MLB are right-handed. In any case, this paper assumes that they do not make a critical difference in the same at-bat.}. On the other hand, the former difference is caused by the psychology of umpires, called omission bias. They tend to avoid judging strikes when two strikes and balls when three balls, which are decisions that end the at-bat without the batter hitting the ball into fair territory.

Figure \ref{fit_gam_02} and \ref{fit_gam_30} illustrate the predicted call probability by batter-handedness (The left-hand side for typical left-handed batters, while the right-hand side is for the typical right-handed batters.). To show that pitch counts dramatically affect the average probabilities, we show two anecdotal cases: 0 balls and 2 strikes (0-2 counts) and 3 balls and 0 strikes (3-0). Graphically, you can see that in 0 - 2 counts, umpires favor batters: less likely to call a strike. Figure \ref{fit_gam_30} suggests the opposite implication in 3-0 counts. These primitive results are consistent with previous literature on the umpires in sabermetrics\footnote{The term "sabermetrics" refers to an attempt to analyze baseball from an objective standpoint using scientific methods and to gain new insights that could not be obtained with the old methodology of analysis based solely on experience and feeling.}.

\begin{figure}[t]
  \begin{tabular}{c}
    %---- 最初の図 ---------------------------
    \begin{minipage}[ht]{.9\hsize}
      \centering
      \caption{Predicted Strike-Call Probability: 0 Balls and 2 Strikes}
      \includegraphics[keepaspectratio, scale = .85]{fig_tab/strike_02.png}
      \label{fit_gam_02}
    \end{minipage} \\
    %---- 2番目の図 --------------------------
    \begin{minipage}[t]{.9\hsize}
      \centering
      \caption{Predicted Strike-Call Probability: 3 Balls and 0 Strikes}
      \includegraphics[keepaspectratio, scale = .85]{fig_tab/strike_30.png}
      \label{fit_gam_30}
    \end{minipage}
    %---- 図はここまで ----------------------
  \end{tabular}
\end{figure}

As will be discussed in Section 6, the estimated probability using GAM has a very large effect on the actual pitch call. In this paper, the sample is divided into 24 subsamples of 12 (0-2 strikes $\times$ 0-3 balls) $\times$ 2, distinguishing between pitch counts and batter left and right, and the probability of each is estimated by GAM and included in the regression in equation (\ref{regression_main}).

\section{Results}

\subsection{Main Result}

This section discusses the effect of bias due to plate umpire on pitch calls.

Initially, we present results for the sample where the plate umpire is from the United States. Table \ref{main_result} shows the results of the estimated equation \ref{regression_main}. The nationalities of the pitcher and batter are listed in rows 3 through 5, with the texts before the underscore (\_) indicating the nationality of the pitcher, and the nationality of the batter follows after it. For example, "NoUS\_US" indicates that the batter is from the United States and the pitcher is from elsewhere. The pitcher's choice and the manager's decision on which pitcher to play against which batter are endogenous variables. Including these variables may be "bad controls," as pointed out by \citet{Angrist2008}. In the first column of Table \ref{main_result}, we show the results with only the estimated strike call probability $\hat{f}$ and nationality information as explanatory variables.

\input{fig_tab/result1.tex}

The estimation results were consistent with the hypothesis. When the pitcher is from the U.S. and the batter is from elsewhere, the plate umpire from the United States tends to decide in favor of the pitcher. Specifically, an average deviation of 0.3-0.4 percentage points from the probability of a strike call estimated solely by the location of the pitch's passage occurs, i.e., the batter is more likely to have a strike called. The estimates were statistically significant (at the 0.1\% level). On the other hand, if the pitcher is from a non-U.S. country and the batter is from the United States, the pitch call will favor the batter. However, the point estimates were relatively small for batters, and the estimation results in column 4, which contained the most explanatory variables, were no longer statistically significant.

The result where both the pitcher and the batter are not from the United States does not support the hypothesis. In this matchup with no intuitive conflict with the umpire the theory predicts to show similar results to the baseline case where both pitcher and batter were from the United States. However, some of the results suggested that the plate umpire would call in favor of the pitcher.

In addition, the results pointed to count-dependent changes in the strike zone and the existence of a hometown bias. These are consistent with the previous results. For example, the percentage of strike calls increases by 0.4 percentage points in the innings table, where most games hit the visiting team. Although these are not the main focus of this study, the fact that they are consistent with previous studies implies the validity of the analysis in this study. In particular, the effect of pitch count is much larger than the ones of player nationality and other factors, with approximately tenfold variation occurring.

\input{fig_tab/result2.tex}

As shown in Figures \ref{fit_gam_02} and \ref{fit_gam_30}, the probability of an average strike call changes rapidly at the edge of the strike zone. The plate umpire is also less sensitive to nationality since a pitch that passes through a position with an extremely high or low average probability is a strike ball for all. Table \ref{edge} shows the estimation results when the sample is restricted to pitches with average strike call probabilities $\hat{f}$ between 30\% and 70\%. This region is called the "shadow zone" and is empirically considered to be the pitches for which the strike zone is likely to change depending on the context.

The results are consistent with our prediction, especially in situations where batters are discriminated against, and the point estimates show 0.8 percentage points to 1.2 percentage points. In other words, the estimates in the full sample are relatively large due to this part of the effect. However, the effect of discrimination was also present for obvious strikes and obvious balls are thrown with an average strike call probability of less than 10 \% or more than 90 \%.

\input{fig_tab/result3.tex}

The last result of the main estimation is about the judgement of the non-American plate umpire. Table \ref{reverse} shows the OLS results with the sample restricted to umpires from countries other than the US. Statistically, the results do not provide strong support for reverse discrimination by minorities. None of the dummy variables for nationality were dominant at the 10\% level. However, the signs of the point estimates are all consistent with the hypotheses, especially when the pitcher is non-American and the batter is American, and the coefficient estimates are relatively large, ranging from 0.3 percentage points to 0.6 percentage points.

These results indicate that plate umpires from the U.S. make pitch calls that are influenced by nationalistic bias based on the country of origin of the pitcher and batter. The effect is particularly large in situations where batters are discriminated against, and they are more likely to underperform themselves by being called more strikes.

\subsection{Size of Effect: Indirect Effects on Players' Salary}

Next, we discuss the average impact of discrimination in size: from Section 6.1. when the plate umpire and pitcher are from the U.S. and the batter is from the rest of the world, the probability of a strike call increases by .4 percentage points per pitch. In this paper, we quantify how much of a financial loss this causes to minority players.

\input{fig_tab/result_4.tex}

Table \ref{salary_regression} quantifies the impact of improvements in player performance metrics on subsequent seasons using data on hitter performance from 2016 to 2018 and annual salary the following year. Specifically, we regress the following equation.

\begin{align}
  W_{i} = \alpha + \beta \text{Nationality}_i + \gamma \text{PerfObs}_{i} + \mathbf{b} \mathbf{X}_{i} \label{sal_reg}
\end{align}

To account for issues such as multi-year contracts and applications for annual salary arbitration, we sum up performance measures and the annual salary data over the three years. $W_{i}$ in equation (\ref{sal_reg}) represents the annual salary over the three years, and $\text{PerfObs}_{i}$ is a proxy variable for the performance measure used by team managers to consider the annual salary for their players. In other words, this value is subject to umpire bias.

$\text{PerfObs}_{i}$ can be further decomposed into the following definition.

\begin{align}
  \text{PerfObs}_{i} = \text{PerfPot}_{i} + \phi \times \text{Attendance}_{i} \label{potential}
\end{align}

$\text{PerfPot}_{i}$ represents the potential, i.e., the true ability of the batter, that would be observed if there were no discrimination by the umpire. $\text{Attendance}_{i}$ represents the number of opportunities distributed to the player. It can be interpreted, in other words, as the frequency with which players who would be subject to discrimination appear on the field. The value of $\Phi$ represents the impact of discrimination as estimated in the main results section and indicates that the more minority batters are plate-appearances, the greater the downward bias on the measure of run creation for that player.

The two performance indicators used in this paper are "Runs Above Replacement (RAR)" and "Bat (Batting)." RAR is a measure of "how many more points a player contributes than a replacement-level\footnote{The replacement level indicates the level of competence of the players in the MLB player market who are being sought, i.e., those whom teams can be immediately acquire by paying minimum guaranteed salary.It will not be explained in detail here. It is sufficient to understand that a player's value can be evaluated in terms of points.} player", and the number can be converted to points per score (runs) in baseball. Therefore, the coefficient value of RAR in equation (\ref{sal_reg}) represents the rate of increase in annual salary when the player's contribution to the team's runs increases by one point (over three years). In short, it stands for the return on contribution. "Bat" is a similar measure of a player's hitting-related ability on a scale of scoring contributions created.

As shown in Equation (\ref{sal_reg}), this regression also includes a proxy variable for the player's home region. This corresponds to the effect of the player's region of origin on the annual salary amount, which remains even after accounting for differences in the player's track record. The estimated coefficients of these dummy variables are statistically significant if there is an effect of discrimination against minority players by team management or consumers (see Figure \ref{fig_path_disc}).

Consider the results in Table \ref{salary_regression}. First, there was no strong support for the possibility of discrimination by team or fan, at least when controlling for factors such as player performance and age. None of the estimates for the regional dummies were statistically significant, or represented treatment that favored minority players. The remainder of the discussion will focus on the effects of discrimination by ballplayers.
Rows 5 and 6 of Table \ref{salary_regression} show the return of runs scored by players relative to their annual salary. It shows that the return on an increase of one point in runs scored by a player over three years is estimated to be about \$224,000, or 2.4\%. The coefficient values are statistically significant (at the 0.1\% level) when using either RAR or Bat, and the point estimates show similar impact size. As mentioned earlier, since these indicators implies a player's ability in the form of runs, quantifying the negative runs creations produced by pitch-calling bias would allow us to measure the impact of the decision in monetary value.

The impact of pitch calls on the creation of runs is done using the concept of Runs Expectancy\footnote{For more information on the concept of expected runs scored, please refer to the Sabermetrics website.} used in MLB.
We can regard baseball as a dynamic game of discrete time in which a pitcher and a batter play against each other with intervals in between. For example, a double by the batter at the beginning of an inning changes the situation from no outs and no runners on base to no outs and no runners on second base. For each state\footnote{Out count: 0, 1, or 2 $\times$ 3 with or without runners on base, 8 ways = 24 different states defined.}, if we average the runs scored going into the rest of the inning, we get the Runs Expectancy table as shown in Table \ref{re24}. For example, with bases loaded, if there are no outs, we can expect to score 2.23 more runs on average in that inning. However, if there is one out, that number decreases to 1.55 runs.In other words, striking out with the bases loaded and no outs create a negative score of 0.68 (1.55 - 2.23). By averaging this change in expected score for example, strikeouts, we can estimate the average (state-independent) run creation value of a strikeout. Similarly, we can calculate how many points the expected score would be reduced (A strike is a decision that disadvantages the batter.) on average if the umpire calls a strike instead of a ball when the batter takes the ball. According to this, the average scoring creation for being taken one strike is about -1.4 points.

\begin{table}[ht]
  \centering
  \caption{Runs Expectancy by 24 States}
  \label{re24}
  \scalebox{1}{
  \begin{tabular}{lrrr}
    \hline
    Outs/Runners\_on & 0 & 1 & 2 \\
    \hline
    \_\_\_ & 0.50 & 0.27 & 0.10 \\     
    \_\_3 & 1.39 & 0.97 & 0.37 \\
    \_2\_ & 1.14 & 0.69 & 0.32 \\
    \_23 & 1.98 & 1.40 & 0.56 \\
    1\_\_ & 0.88 & 0.52 & 0.22 \\
    1\_3 & 1.78 & 1.21 & 0.49 \\
    12\_ & 1.46 & 0.93 & 0.43 \\
    123 & 2.23 & 1.55 & 0.76 \\ \hline
  \end{tabular}}
\end{table}

Minority batters are on average 0.4 percentage points more likely to have a strike called per pitch, so for example, 100 pitches against an American-born player with an American-born umpire would score -1.4 points $\times$ 0.004 percent per pitch $\times$ 100 pitches = 0.56 points. It will harm the observed ability to runs creation.
Figure \ref{dist_lost} quantifies the sum of the runs created lost over three years by minority players for whom Salary data are available. Since players with fewer opportunities to bat are also less likely to be discriminated against, many players appear to have little impact on the total over the three years. On the other hand, hitters who are the core players of their teams and bat all year long have more opportunities to bat in situations where they are discriminated against, which means that they lose nearly two runs in three years due to discriminatory pitch calls. In particular, the best players in the top 25 percentile among the minorities lost an average of 0.58 points worth of runs. As shown in Table \ref{salary_regression}, the return per runs created is \$224,000, so the estimated salary they lose over three years is about \$130,000. Some of the players signed before free agency sign the contract with about the minimum compensation set at about \$570,000. This is a loss that we should not ignore.

\begin{figure}[ht]
  \centering
  \caption{Distribution of the Negative Contribution by the Biased Calls}
  \includegraphics[scale = .8]{fig_tab/dist_lost.png}
  \label{dist_lost}
\end{figure}

\subsection{Endogeneity: Bahavioral Response of the Batters}

The impact size discussed in the two sections is a lower bound. This is because, in the context of the strike zone, significant endogeneity is noted.

If minority hitters are empirically (or quantitatively) aware of the existence of discrimination by plate umpires, they will decide whether or not to swing at a pitch given a wider strike zone.

\begin{align}
  U_s = & E[\text{Runsvalue} | \text{Contact}] \times Pr(\text{Contact}) + E[\text{Runsvalue} | \text{Missed}] \times (1 - Pr(\text{Contact}))\label{ut_swing} \\
  U_t = & E[\text{Runsvalue} | \text{Called Strike}] \times Pr(\text{Called Strike}) \notag \\
  & + E[\text{Runsvalue} | \text{Called Ball}] \times (1 - Pr(\text{Called Strike})) \label{ut_take}
\end{align}

The batter's decision to swing or take can be described by the following utility function The expressions (\ref{ut_swing}) and (\ref{ut_take}) represent the incremental scoring contribution created when the batter swings at the pitch, respectively. When a swing is made, if the batter contacts the ball, the expected value of the contribution created by the generated pitch is generated, while a loss of one strike always occurs if the batter swings and misses. On the other hand, if the pitch is taken, it is determined whether the pitch is a ball or a strike based on the average strike call probability $\hat{f}$ estimated by \ref{logit}. Taking convex combination of the value of one strike and one ball by the average probability of a strike for the pitch location of that pitch, we can calculate the expected value of taking a pitch as measured by scoring creation.

Therefore, the above two equations can be rewritten as a function of pitch location $(\text{plate}\_x, \text{plate}\_z)$ as follows.

\begin{align}
  U_s = & S(\text{plate}\_x, \text{plate}\_z) \label{swingvalue} \\
  U_t = & E[\text{Runsvalue} | \text{Called Strike}] \times \hat{f}(\text{plate}\_x, \text{plate}\_z)\notag \\
  & + E[\text{Runsvalue} | \text{Called Ball}] \times (1 - \hat{f}(\text{plate}\_x, \text{plate}\_z)) \label{takingvalue}
\end{align}

Table \ref{delta_rv} shows the variant of the average expected score when a pitch was a strike, a ball, or a foul in each pitch count. Substituting these values into the equation (\ref{takingvalue}) calculates the average value when a pitch is taken.

For example, consider the case of taking a pitch from 0 balls and 2 strikes (e.g. "3-2" represents 3 balls and 2 strikes) such that $\hat{f} = 50\%$. If the pitch is ruled a strike, the value is -0.164; if it is a ball, it is 0.043. Thus, the value of taking this pitch is

\[-0.164 \times 0.5 + 0.043 \times (1 - 0.5) = -0.0605. \]

\begin{table}[ht]
  \centering
  \caption{$\Delta$ Run Expectation at Each Pitch Count}
  \label{delta_rv}
  \scalebox{1}{
  \begin{tabular}{lrrr}
    \hline
    Count & if strike & if ball & if foul \\ \hline
    0-0  &  -0.042  & 0.037 & -0.042 \\
    1-0   & -0.055  & 0.060 & -0.055 \\
    2-0  &  -0.079  & 0.097 & -0.079 \\
    3-0  &  -0.083  & 0.116 & -0.083 \\
    0-1  &  -0.059  & 0.024 & -0.059 \\
    1-1  &  -0.040  & 0.036 & -0.040 \\
    2-1  &  -0.042  & 0.093 & -0.042 \\
    3-1  &  -0.003 & 0.200 & -0.003\\
    0-2  &  -0.164  &  0.043 & 0      \\
    1-2  &  -0.207  &  0.034 & 0      \\
    2-2  &  -0.241  &  0.132  & 0      \\
    3-2  &  -0.373  &  0.203  & 0\\ \hline
  \end{tabular}}
\end{table}

In a similar way to finding $\hat{f}$, we use the expected value of run creation when a batter swings at a pitch that has passed a specific pitch location, the function $S(. ,.) $ can be calculated.

\begin{align}
  S(\text{plate\_x}, \text{plate\_z}) & = r(\text{plate\_x}) + r(\text{plate\_x}, \text{plate\_z}) + r(\text{plate\_z}) \label{spline_swing}
\end{align}

$r(., .)$ is the spline regression.

\input{fig_tab/runvalues.tex}

If the batter strikes out on a pitch, the value of "if strike" in the Table \ref{delta_rv} is used as the explained variable according to the pitch count. On the other hand, if the batter hits a double, for example, the average scoring value of a double, 0.76, is the value of the swing. Table \ref{rv_plays} shows the average value that each play has.

Using these values, we can quantify the extent to which incentives exist to swing at a ball pitched at a particular pitch location. For a pitch $i$,

\begin{align}
  \text{Incentive}_i = & S(\text{plate\_x}, \text{plate\_z}) - \hat{f}(\text{plate}\_x, \text{plate}\_z) \times \text{Value when Strike}_i \notag \\
  & - (1 - \hat{f}(\text{plate}\_x, \text{plate}\_z)) \times \text{Value when Ball}_i
\end{align} 

This corresponds to the difference between the equations (\ref{swingvalue}) and (\ref{takingvalue}), where the larger the number, the larger the expected value of runs created by swinging at a pitch rather than taking it. Figure \ref{swinging_values_illustration} illustrates these values given pitch locations, seperating the handedness of the pitchers and the batters. There are differences depending on the pitcher's or batter's handedness, but in general, you will find that the expected value is higher when you swing around the heart of the strike zone.

\begin{figure}[ht]
  \centering
  \caption{Expected Run Values when Swinging}
  \includegraphics[scale = 1]{fig_tab/values_when_swing_updated.png}
  \label{swinging_values_illustration}
\end{figure}

Using this, we estimate the following regression.

\begin{align}
  \mathbf{1}(\text{Swung})_i = \beta_1 + \psi \text{Nationality}_i + \beta_2 \text{Incentive}_i + \beta_4 \mathbf{X}
\end{align}

$\text{Nationality}_i$ corresponds to the nationality of the player and plate umpire, as in equation (\ref{regression_main}). The dependent variable is the dummy that indicates if the batter swings or not. If minority players are aware of the existence of discrimination and expect their probability of calling a strike to increase when facing an American pitcher, they will swing more even in situations where the incentive to swing is relatively small.

\input{fig_tab/result01.tex}

Table \ref{behavioral_responses} shows the estimation results for OLS, where columns 1 and 2 include all pitches in the sample, and columns 3 and 4 restrict the sample to the "shadow zone" where the percentage of strike calls predicted by the location of the pitch's path is between 30\% and 70\%. Since the sample is limited to cases where a minority batter is at bat, there are three nationality states: the pitcher is from the US, the plate umpire is from the US, and both of them are from the US. The estimation results did not support the existence of a change in batter behavior according to nationality. The value of the point estimate varied above and below zero depending on the sample selected, and when all pitches were included in the sample, it even resulted in a rather low percentage of swings in situations where he himself was discriminated against.

\section{Discussion}

The probability variation of strike calls by nationality bias was generally as predicted by the theory. Many studies dealing with the strike zone context, including \citet{Parsons_etal_2011}, have used the race of the player as a proxy variable instead of nationality. On the other hand, for example, \citet{Sandberg2018} pointed out the existence of in-group bias by the nationality of the umpire using a dressage dataset. Since some studies in the context of the strike zone use country-of-origin-dependent Latino/Asian identification and lists of black athletes in major media articles as a way to identify their race, whether the bias is due to nationality or race There is room for debate as to whether the bias is due to nationality or race. On the other hand, in MLB games, the names of the players at bat and the pitchers on the mound are called out by speakers in the stadium, and their names are written on the scoreboard attached to the batter's eye. In this situation, the umpire can know their names at any time. Using data from their experiments, \citet{Bertrand_et_al_2004} and \citet{Edelman_et_al_2017} have pointed out that people's names may be used as cues to determine their race. In addition, the plate umpire, in particular, is located at a very close distance to the player and can easily know the skin color of the player. \citet{Monk2015}and others have argued that people's skin color predicts the degree of discrimination they are subjected to, and intuitively, these effects are likely to occur in baseball. Since the faces of athletes are available to the public, further research in this area using such data may provide insights into discriminating between nationality bias and racial discrimination.

\section{Conclusion}

This paper has used the context of strike calls in the MLB to show evidence for minority discrimination by evaluators (umpires). The in-group bias that occurs when evaluating the achievements of workers (players) also causes workers to suffer economic losses through subjective evaluations by employers (team managers) who enter into contracts with workers based on the bias. This paper suggests that even though umpires are highly skilled professionals, they are affected by their in-group bias. Such a problem can occur even in a more general work environment that lacks objective measures of worker performance and so we can consider our results as a contribution to the labor economics literature using empirical data.

Another advantage of research using MLB data is that it allows us to track the behavior of workers discriminated against, as discussed in Section 6.3. According to the results, batters discriminated against would recognize it and make behavioral changes, while the financial losses caused by discrimination were very severe. The fact that the effects of such bias exist in the MLB labor market, which attracts players from all over the world, may shed some light on the debate over the introduction of non-human entities such as robot umpires.

%-------Refs not cited--------------

\nocite{Parsons_etal_2011}
\nocite{Tainsky_etal_2015}
\nocite{Agan_2018}
\nocite{Ashenfelter1987}
\nocite{Archsmith2021}
\nocite{baseballsavant}
\nocite{bbref}
\nocite{fg}
\nocite{chadwick}
\nocite{retrosheets}
\nocite{Angrist2014}
\nocite{Fiva2016}
\nocite{Jiang2019}
\nocite{MCCORMICK2001201}
\nocite{Kim2014}
\nocite{jiang_qian_yonker_2019}
\nocite{chen_et_al_2016}
\nocite{mlbam}
%-----------------------------------


\bibliography{C:/texlive/texmf-local/bibtex/bib/local/myrefs}
%\bibliography{refs}

\end{document}
